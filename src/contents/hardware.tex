\section{The Hardware Underpinnings of SGX}

The security of SGX hinges on the assumption that it is very difficult for an
attacker to produce an attestation that contains attacker-supplied information
in the enclave field. In order to evaluate this claim, we must understand the
hardware protection mechanisms and the encryption schemes involved in the
process.

Most of the material in this section comes from Intel's
patents~\cite{intel2013patent1, intel2013patent2}, because the SGX papers and
reference consider this information an implementation detail. We point out when
a piece of information matches the SGX manual or one of the SGX papers.

According to the Intel patents, the SGX instructions are implemented in
microcode. The statement in \cite{ananti2013sgx} that a microcode update
changes the platform TCB confirms this finding. The patents state that SGX
requires very few hardware changes, and most of the implementation is in
microcode, as a positive fact, fueling our suspicion that minimizing hardware
changes was a high priority in the SGX design.


\subsection{Memory Protection Mechanisms}

The SGX reference manual states that the Processor Reserved Memory (PRM) is
allocated by the BIOS, by setting the PRM range registers (PRMRR). The patents
refer to this range as the secure enclave range registers (SERR), and state
that the memory range is protected from DMA accesses by a dedicated pair of
entries in the memory controller's SAD (System Address Decoder) and TAD
(Target Address Decoder), which are briefly described in \cite{ziakas2010qpi}.
The patents state that the SAD and TAD entries mirror the PRMRR registers
which, if done correctly, does prevent against DMA snooping.

The SGX reference manual states that the EPC memory is protected against
physical snooping attacks on the DRAM by an implementation-dependent mechanism,
and suggests that the most likely implementation is a Memory Encryption Engine
(MEE). The MEE is called a crypto engine in the Intel patents, which state that
the crypto engine is connected to the QuickPath Interconnect (QPI) home agent
in each processor, and it encrypts all RAM accesses in a range called the
Crypto Memory Aperture (CMA), before they reach the memory controller. The CMA
range is configured by setting the Crypto Memory Range Registers (CMRR). The
CMA contains the EPC, EPCM, and other data structured used by the SGX
implementations, which leads us to believe that the CMA covers the entire PRM
range, and the CMRR and PRMRR are identical. One of the SGX
papers~\cite{mckeen2013sgx} states that the PRM may be covered by one or more
MEE regions.

The data inside the CMA is lost when the CPU is powered down (including when it
enters the S3 power management state), so enclaves must be either torn down by
the OS, or their EPC pages must be evicted to RAM, and eventually to disk.

The EPC encryption described above is only intended to defend against physical
attacks. Enclaves are isolated from system software (the OS and hypervisor) by
access control checks in the CPU, as described in \S~\ref{sec:epcm}. The
patents specify the protection algorithm with more clarity.

The patents also specify that the memory access checks are performed in the
Page-Miss Handler (PMH) microcode, which is invoked during TLB misses. The PMH
performs the SGX access checks described below. If the access checks succeed,
the PMH creates a TLB entry, as it normally would. If the access checks fail,
the PMH aborts, which results in a Page Fault. The desire to restrict SGX
access checks to the PMH introduces the requirement to flush a logical
processor's TLBs when it enters or exits enclave mode. This is accomplished by
adding 1 bit to TLB entries, which differentiates between enclave pages and
non-enclave pages. On enclave entry and exit, all the TLB entries that have the
enclave bit set are flushed. One of the SGX papers~\cite{mckeen2013sgx}
confirms that the EPCM is checked by the PMH. The Intel patents state that
future implementations may optimize away the TLB flushes by adding enclave ID
tags to TLB entries.

The SGX access checks occur after the normal page address translation process
(described in the Intel Software Developer Manual~\cite{intel2014manual})
completes. If the resulting physical address does not fall within the CMA, the
SGX access checks do not apply. A CMA memory access is only allowed if
originates from the CPU's microcode (used to implement the SGX instructions),
or if it targets an EPC page, and the logical processor performing the access
is executing the enclave that owns the page. This is checked by the following
steps:

\begin{enumerate}
  \item the physical address must fall in an EPC range
  \item the page's EPCM entry must have the present (P) bit set, and not have
        the blocked (B) bit set
  \item the current enclave's ID must match the enclave ID in the page's EPCM
        entry
  \item the linear address used to access the page must match the one in the
        page's EPCM entry
\end{enumerate}


The Intel patents call ELRANGE the Enclave Linear Space (ELS) range.



\subsection{Key Hierarchy and Derivation}

According to Intel's patents, the SGX implementation relies on a complex key
derivation process rooted on global secret keys in the CPU circuitry, and on
secrets embedded in the processor's eFUSEs. eFUSE information can be extracted
efficiently (Chipworks quoted us \$50-250k for extracting the entire eFUSE
contents from an Intel i5 processor), so some of the eFUSE secrets are
encrypted with a master key (referred to as a ``global wrapping logic key'' in
the patents).

The patents state that encrypting the eFUSE secrets by the logic key makes them
harder to extract via hardware monitoring tools, and protects them while in
transit to the CPU during the manufacturing process. This assumes that it is
very expensive to obtain the global key from a CPU, by virtue of the low
feature size.

The SGX patents describe two ``logic keys'' embedded in the CPU's circuitry,
which are the same for all CPUs in a stepping, making them essentially global
keys. The \textit{global wrapping logic key} is a 128-bit AES key, and it is
used to encrypt a subset a 256-bit A.x value used to re-create the CPU's EPID
key, and a 128-bit \textit{pre-seed key 0}. The eFUSEs also contain a 128-bit
\textit{pre-seed key 1} and a 32-bit EPID group ID, which are stored in
cleartext.

The SGX manuals~\cite{intel2013sgxmanual, intel2014sgx2manual} mention a
16-byte CPU security version number (SVN), which contains the version numbers
of various TCB components, and is a source in the key derivation process. The
patents further specify that the SVN register is made up of (most likely 8-bit)
sections that contain the SVNs of each layer in the SGX initialization process,
and that each initialization step sets the corresponding section to its SVN,
and then locks it for the duration of the power-up cycle.

Intel's patents disclose that the key derivation process uses 128-bit AES in
ECB mode as a pseudo-random function (PRF). When an SVN is an input to a key
derivation process, a \textit{PRF loop} is used, where the PRF is applied to
a constant



\cite{anati2013sgx} confirms that the attestation uses Intel's
EPID~\cite{brickell2009epid} group signature scheme.


Key Request Inputs

\begin{table}[hbt]
  \center{\begin{tabularx}{\columnwidth}{| l | l | X |}
  \hline
  \textbf{Name} & \textbf{Size} & \textbf{Description}\\
  \hline
  Key Name & 2 & Which key will be derived \\
  \hline
  Policy & 2 & Whether the key derivation uses MRENCLAVE or MRSIGNER \\
  \hline
  ISV SVN & 2 & Developer-assigned SVN \\
  \hline
  CPU SVN & 16 & 128-bit TCB SVN \\
  \hline
  KeyID & 32 & Used for wear-out protection \\
  \hline
  Attribute Mask & 16 & Which SECS ATTRIBUTES are used in Seal Key \\
  \hline
  MISC Mask & 2 & Which SECS MISCSELECT bits are used in Seal Key \\
  \hline
  \end{tabularx}}
  \caption{Values of the PT (page type) field in an EPCM entry.}
  \label{fig:key_request_inputs}
\end{table}

Key names: LAUNCH, PROVISION, PROVISION\_SEAL, REPORT, SEAL.


\subsection{Data Structures}

The Intel patents state that the 128-bit enclave key is stored in the enclave's
SECS page, and that SGX security depends on enclaves not being able to read
their own SECS pages. They also state that the enclave key is used to encrypt
the EPC pages when they are evicted to untrusted RAM. The patents also state
that the enclave's SECS page contains the SVN of the launch permit creator.

The Intel patents state that each TCS has two fields that store the values of
the DR7 and IA32\_DEBUGCTL registers during EENTER, and are used to restore the
original values during EEXIT and AEX.

The Intel patents state that the enclave ID used in EPCM entries is the
phyisical address of the page holding the enclave's SECS, without the bottom
12 bits that are guaranteed to be zero. The SGX manual indicates that when an
EPC page is allocated via EADD, the physical address of its enclave's SECS is
cached, until the page is removed via EREMOVE. This, together with the fact
that the enclave ID's location in the SECS is not specified, indicates that the
enclave's ID is indeed the physical address of its SECS page.


The Intel patents indicate that EREPORT's KeyID is initialized to a random
value on each processor power cycle, and is incremented after $2^{32}$ AES
operations that use the value. They also indicate that each EREPORT may
increment the KeyID by 1.


