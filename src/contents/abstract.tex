\begin{abstract}

Intel's Software Guard Extensions (SGX) is a set of extensions to the Intel
architecture that aims to provide integrity and privacy guarantees to
security-sensitive computation performed on a computer where all the privileged
software (kernel, hypervisor, etc) is potentially malicious.

This paper analyzes Intel SGX, based on the 3
papers~\cite{mckeen2013sgx, anati2013sgx, hoekstra2013sgx} that introduced it,
on its reference manual~\cite{intel2014sgx2manual}, and on two patent
applications~\cite{intel2013patent1, intel2013patent2}. We use the papers and
reference manual as primary data sources, and only draw on the patents to fill
in missing information.

We explain the threat model and mechanisms used by SGX and analyze their
security properties. In conclusion, we agree with the Intel authors that SGX
protects the integrity of sensitive computation, and provides some privacy
guarantees. We show straight-forward methods for obtaining the memory access
patterns in an SGX program. We argue that memory access pattern leaks can allow
an adversary to learn private information, and analyze the limitations of SGX
from this
perspective.

\end{abstract}
