\subsection{The Bastion Architecture}

The Bastion architecture \cite{champagne2010bastion} introduced the use of a
trusted hypervisor to provide secure containers to applications running inside
unmodified, untrusted operating systems. Bastion's hypervisor ensures that the
operating system does not interfere with the secure containers. We only
describe Bastion's virtualization extensions to architectures that use nested
page tables, like Intel's VMX \cite{uhlig2005vmx}.

The hypervisor enforces the containers' desired memory mappings in the OS page
tables, as follows. Each Bastion container has a Security Segment that lists
the virtual addresses and permissions of all the container's pages, and the
hypervisor maintains a Module State Table that stores an inverted page map,
associating each physical memory page to its container and virtual address. The
processor's hardware page walker is modified to invoke the hypervisor on every
TLB miss, before updating the TLB with the address translation result. The
hypervisor checks that the virtual address used by the translation matches the
expected virtual address associated with the physical address in the Module
State Table.

Bastion's cache lines are not tagged with container identifiers. Instead, only
TLB entries are tagged. The hypervisor's TLB miss handler sets the container
identifier for each TLB entry as it is created. Similarly to XOM and Aegis, the
secure processor checks the TLB tag against the current container's identifier
on every memory access.

Bastion offers the same protection against physical DRAM attacks as Aegis,
without the restriction that a container's data must be stored inside a
continuous DRAM range. This is accomplished by extending cache lines and TLB
entries with flags that enable memory encryption and HMACing. The hypervisor's
TLB miss handler sets the flags on TLB entries, and the flags are propagated to
cache lines on memory writes.

The Bastion hypervisor allows the untrusted operating system to evict secure
container pages. The evicted pages are encrypted, HMACed, and covered by a
Merkle tree maintained by the hypervisor. Thus, the hypervisor ensures the
privacy, authenticity, and freshness of the swapped pages. However, the ability
to freely evict container pages allows a malicious OS to learn a container's
memory accesses with page granularity. Furthermore, Bastion's threat model
excludes cache timing attacks.

Bastion does not trust the platform's firmware, and computes the cryptographic
hash of the hypervisor after the firmware finishes playing its part in the
booting process. The hypervisor's hash is included in the measurement reported
by software attestation.
