\HeadingLevelB{Software Attacks on Peripherals}
\label{sec:device_attacks}

Threat models for secure architectures generally only consider software attacks
that directly target other components in the software stack running on the CPU.
This assumption results in security arguments with the very desirable property
of not depending on implementation details, such as the structure of the
motherboard hosting the processor chip.

The threat models mentioned above must classify attacks from other motherboard
components as physical attacks. Unfortunately, these models would mis-classify
all the attacks described in this section, which can be carried out solely by
executing software on the victim processor. The incorrect classification
matters in cloud computing scenarios, where physical attacks are significantly
more expensive than software attacks.


\HeadingLevelC{PCI Express Attacks}
\label{sec:pcie_attacks}

The PCIe bus~(\S~\ref{sec:motherboard}) allows any device connected to the bus
to perform \textit{Direct Memory Access}~(DMA), reading from and writing to
the computer's DRAM without the involvement of a CPU core. Each device is
assigned a range of DRAM addresses via a standard PCI configuration mechanism,
but the bus does not restrict a device from performing DMA targeting DRAM
addresses outside its assigned range.

Without any additional protection mechanism, an attacker that compromises system
software can take advantage of programmable devices to access any DRAM region,
yielding capabilities that were traditionally associated with a DRAM bus tap.
For example, an early implementation of Intel TXT~\cite{grawrock2009txt} was
compromised by programming a PCIe NIC to read TXT-reserved DRAM via DMA
transfers~\cite{wojtczuk2011txt}. Recent versions have addressed this attack by
adding extra security checks in the DMA bus arbiter.
\S~\ref{sec:sgx_related_intel_txt} provides a more detailed description of
Intel TXT.


\HeadingLevelC{DRAM Attacks}
\label{sec:rowhammer_attack}

The rowhammer DRAM bit-flipping attack~\cite{kim2014rowhammer,
google2015rowhammer, gruss2015rowhammer} is an example of a different class of
software attacks that exploit design defects in the computer's hardware.
Rowhammer took advantage of the fact that some mobile DRAM
chips~(\S~\ref{sec:motherboard}) refreshed the DRAM's contents slowly enough
that repeatedly changing the contents of a memory cell could impact the charge
stored in a neighboring cell, which resulted in changing the bit value obtained
from reading the cell. By carefully targeting specific memory addresses, the
attackers caused bit flips in the page tables used by the CPU's address
translation (\S~\ref{sec:paging}) mechanism, and in other data structures used
to make security decisions.

The issue exploited by the rowhammer attack most likely stems from an incorrect
design assumption. The DRAM engineers probably only thought of non-malicious
software and assumed that an individual DRAM cell cannot be accessed too
often, as repeated accesses to the same memory address would be absorbed by the
CPU's caches (\S~\ref{sec:caching}). However, malicious software can take
advantage of the \texttt{CLFLUSH} instruction, which flushes the cache line
that contains a given DRAM address. \texttt{CLFLUSH} is intended as a method
for applications to extract more performance out of the cache hierarchy, and
is therefore available to software running at all privilege levels. Rowhammer
exploited the interaction between \texttt{CLFLUSH}'s availability and the DRAM
engineers' invalid assumptions to obtain capabilities that are normally
associated with an active DRAM bus attack.


\HeadingLevelC{The Performance Monitoring Side Channel}
\label{sec:perfmon_attacks}

% On Die Digital Thermal Sensors: SDM S 14.7.5

Intel's \textit{Software Development Manual}~(SDM)~\cite{intel2015sdm} and
Optimization Reference Manual~\cite{intel2014optimization} describe a vast
array of performance monitoring events exposed by recent Intel processors,
such as branch mispredictions (\S~\ref{sec:out_of_order}). The SDM also
describes digital temperature sensors embedded in each CPU core, whose readings
are exposed using Model-Specific Registers~(MSRs)~(\S~\ref{sec:address_spaces})
that can be read by system software.

An attacker that compromises a computer's system software and gains access to
the performance monitoring events or the temperature sensors can obtain the
information needed to carry out a power analysis attack, which normally
requires physical access to the victim computer and specialized equipment.


\HeadingLevelC{Attacks on the Boot Firmware and Intel ME}
\label{sec:firmware_attacks}

Virtually all motherboards store the firmware used to boot the computer in a
flash memory chip (\S~\ref{sec:motherboard}) that can be written by system
software. This implementation strategy provides an inexpensive avenue for
deploying firmware bug fixes. At the same time, an attack that compromises the
system software can subvert the firmware update mechanism to inject malicious
code into the firmware. The malicious code can be used to carry out a cold boot
attack, which is typically considered a physical attack. Furthermore, malicious
firmware can run code at the highest software privilege level, System
Management Mode (SMM, \S~\ref{sec:rings}). Last, malicious firmware can modify
the system software as it is loaded during the boot process. These avenues give
the attacker capabilities that have traditionally been associated with DRAM bus
tapping attacks.

The Intel Management Engine (ME)~\cite{ruan2014intelme} loads its firmware
from the same flash memory chip as the main computer, which opens up the
possibility of compromising its firmware. Due to its vast management
capabilities (\S~\ref{sec:intel_me}), a compromised ME would bring most of the
powers that come with installing active probes on the DRAM bus, the PCI bus,
and the System Management bus (SMBus), as well as power consumption meters.
Thanks to its direct access to the motherboard's Ethernet PHY, the probe would
be able to communicate with the attacker while the computer is in the Soft-Off
state, also known as S5, where the computer is mostly powered off, but is still
connected to a power source.  The ME has significantly less computational power
than probe equipment, however, as it uses low-power embedded components, such
as a 200-400MHz execution core, and about 600KB of internal RAM.

The computer and ME firmware are protected by a few security measures. The
first line of defense is a security check in the firmware's update service,
which only accepts firmware updates that have been digitally signed by a
manufacturer key that is hard-coded in the firmware. This protection can be
circumvented with relative ease by foregoing the firmware's update services,
and instead accessing the flash memory chip directly, via the PCH's SPI bus
controller.

The deeper, more powerful, lines of defense against firmware attacks are rooted
in the CPU and ME's hardware. The bootloader in the ME's ROM will only load
flash firmware that contains a correct signature generated by a specific Intel
RSA key. The ME's boot ROM contains the SHA-256 cryptographic hash of the RSA
public key, and uses it to validate the full Intel public key stored in the
signature. Similarly, the microcode bootstrap process in recent CPUs will only
execute firmware in an Authenticated Code Module
(ACM, \S~\ref{sec:uefi_sec_details}) signed by an Intel key whose SHA-256 hash
is hard-coded in the microcode ROM.

However, both the computer firmware security checks \cite{wojtczuk2010bios,
furtak2014bios} and the ME security checks \cite{tereshkin2009amt} have been
subverted in the past. While the approaches described above are theoretically
sound, the intricate details and complex interactions in Intel-based systems
make it very likely that security vulnerabilities will creep into
implementations. Further proving this point, a security
analysis~\cite{ververis2010security} found that early versions of Intel's
Active Management Technology~(AMT), the flagship ME application, contained an
assortment of security issues that allowed an attacker to completely take over
a computer whose ME firmware contained the AMT application.


\HeadingLevelC{Accounting for Software Attacks on Peripherals}

The attacks described in this section show that a system whose threat model
assumes no software attacks must be designed with an understanding of all the
system's buses, the programmable devices that may be attached to them. The
system's security analysis must argue that the devices will not be used in
physical-like attacks. The argument will rely on barriers that prevent
untrusted software running on the CPU from communicating with other
programmable devices, and on barriers that prevent compromised programmable
devices from tampering with sensitive buses or DRAM.

Unfortunately, the ME, PCH and DMI are Intel-proprietary and largely
undocumented, so we cannot assess the security of the measures set in place to
protect the ME from being compromised, and we cannot reason about the impact
of a compromised ME that runs malicious software.
