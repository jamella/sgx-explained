\subsection{Cache Timing Attacks}
\label{sec:cache_timing}

% TODO: Figure out how to mention MSR's attacks that collect page fault
%       addresses.

Cache timing attacks~\cite{banescu2011cache} are a powerful class of software
attacks that can be mounted entirely by application
(ring 3, \S~\ref{sec:rings}) code. The attacker software measures the latency
of accesses to its own memory in order to determine whether the accesses caused
misses in a cache that is shared with a victim program. The memory accesses are
carefully chosen to reveal the memory access pattern of the victim program.
Cache timing attacks do not access the victim's memory directly, so they are
\emph{not} prevented by the address translation-based protections
(\S~\ref{sec:paging}) implemented in today's kernels and hypervisors.

Cache timing attacks are known to retrieve cryptographic keys used by
AES~\cite{bonneau2006aes}, RSA~\cite{brumley2005rsa},
Diffie-Hellman~\cite{kocher1996timing}, and elliptic-curve
cryptography~\cite{brumley2011ecc}.
Early attacks required access to the victim's CPU core, but more sophisticated
recent attacks~\cite{yarom2013llctiming, liu2015llctiming} are able to use the
last-level cache (LLC), which is shared by all the cores on a CPU. Recently,
cache-timing attacks were demonstrated to be mountable via JavaScript code in a
page visited by a Web browser~\cite{oren2015jstiming}.

Given this pattern of vulnerabilities, ignoring cache timing attacks is
dangerously similar to ignoring the string of demonstrated attacks which led to
the deprecation of SHA-1~\cite{nist2014sha1policy, google2014sha1deprecation,
microsoft2014sha1deprecation}.
