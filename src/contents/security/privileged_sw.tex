\subsection{Privileged Software Attacks}
\label{sec:system_software_attacks}

The rest of this section points to successful exploits that execute at each of
the privilege levels described in \S~\ref{sec:rings}, motivating the SGX design
decision to assume that all the privileged software on the computer is
malicious. \cite{rutkowska2015intelsux} describes all the programmable hardware
inside Intel computers, and outlines the security implications of compromising
the software running it.

SMM, the most privileged execution level, is only used to handle a specific
kind of interrupts (\S~\ref{sec:interrupts}), namely
\textit{System Management Interrupts} (SMI). SMIs were initially designed
exclusively for hardware use, and were only triggered by asserting a dedicated
pin (SMI\#) in the CPU's chip package. However, in modern systems, system
software can generate an SMI by using the LAPIC's IPI mechanism. This opens up
the avenue for SMM-based software exploits.

% System Management Mode: SDM S 34
% SMRAM: SDM S 34.4
% SMRAM Caching: SDM S 34.4.2

The SMM handler is stored in  \textit{System Management RAM} (SMRAM) which, in
theory, is not accessible when the processor isn't running in SMM. However, its
protection mechanisms were bypassed multiple times~\cite{duflot2006smm,
rutkowska2008remap, wojtczuk2009smm, kallenberg2014smm}, and SMM-based
rootkits~\cite{wecherowski2009smm, embleton2010smm} have been demonstrated.
Compromising the SMM grants an attacker access to all the software on the
computer, as SMM is the most privileged execution mode.

Xen \cite{zhang2008xen} is a very popular representative of the family of
hypervisors that run in VMX root mode and use hardware virtualization. At
150,000 lines of code~\cite{xen2015loc}, Xen's codebase is relatively small,
especially when compared to a kernel. However, Xen still has had over 40
security vulnerabilities patched in \textbf{each} of the last three years
(2012-2014) \cite{cvedetails2014xen}.

\cite{mccune2010trustvisor} proposes using a very small hypervisor together
with Intel TXT's dynamic root of trust for measurement (DRTM) to implement
trusted execution. \cite{vasudevan2010requirements} argues that a dynamic root
of trust mechanism, like Intel TXT, is necessary to ensure a hypervisor's
integrity.  Unfortunately, the TXT design requires an implementation complex
enough that exploitable security vulnerabilities have creeped in
\cite{wojtczuk2009txt2, wojtczuk2011txt}. Furthermore, any SMM attack can be
used to compromise TXT \cite{wojtczuk2009txt}.

The monolithic kernel design leads to many opportunities for security
vulnerabilities in kernel code. Linux is by far the most popular kernel for
IaaS cloud environments. Linux has \emph{17 million} lines of
code~\cite{anthony2014linuxsize}, and  has had over 100 security
vulnerabilities patched in \textbf{each} of the last three years
(2012-2014)~\cite{cvedetails2014linux, chen2011linux}.
