\subsection{Physical Attacks}
\label{sec:physical_attacks}

Physical attacks are generally classified according to their cost, which
factors in the equipment needed to carry out the attack and the attack's
complexity. Joe Grand's DefCon presentation~\cite{grand2004physicalattacks}
provides a good overview with a large number of intuition-building pictures.

The simplest type of physical attack is a denial of service attack performed by
disconnecting the victim computer's power supply or network cable. The threat
models of most secure architectures ignore this attack, because denial of
service can also be achieved by software attacks that compromise the computer's
system software, such as the hypervisor.


\subsubsection{Port Attacks}

Slightly more involved attacks rely on connecting a device to an existing port
on the victim computer's case or motherboard~(\S~\ref{sec:motherboard}). A
simple example is a \textit{cold boot attack}, where the attacker plugs in a
USB flash drive into the victim's case and causes the computer to boot from
the flash drive, whose malicious system software receives unrestricted access
to the computer's peripherals.

More expensive physical attacks that still require relatively little effort
target the debug ports of various peripherals. The cost of these attacks is
generally dominated by the expense of acquiring the development kits needed to
connect to the debug ports. For example, recent Intel processors include the
Generic Debug eXternal Connection~(GDXC)~\cite{yuffe2011sandybridge,
intel2011gdxc}, which collects and filters the data transfered by the uncore's
ring bus (\S~\ref{sec:cache_coherence}), and reports it to an external
debugger.

The threat models of secure architectures generally ignore debug port attacks,
under the assumption that devices sold for general consumption have their debug
ports irreversibly disabled. In practice, manufacturers have strong incentives
to preserve debugging ports in production hardware, as this facilitates the
diagnosis and repair of defective units. Due to insufficient documentation on
this topic, we ignore the possibility of GDXC-based attacks. Fortunately, a
recent Intel patent~\cite{shanbhogue2015gdxcsgx} indicates that Intel engineers
are tackling at least some classes of attacks targeting debugging ports.


\subsubsection{Bus Tapping Attacks}

More complex physical attacks consist of installing a device that taps a bus on
the computer's motherboard (\S~\ref{sec:motherboard}). \textit{Passive attacks}
are limited to monitoring the bus traffic, whereas \textit{active attacks} can
modify the traffic, or even place new commands on the bus. \textit{Replay
attacks} are a notoriously difficult to defeat class of active attacks, where
the attacker first records the bus traffic, and then selectively replays a
subset of the traffic. Replay attacks bypass systems that rely on static
signatures or HMACs, and generally aim to double-spend a limited resource.

The cost of bus tapping attacks is generally dominated by the cost of the
equipment used to tap the bus, which increases with bus speed and complexity.
For example, the flash chip that stores the computer's firmware is connected to
the PCH via an SPI bus (\S~\ref{sec:motherboard}), which is simpler and much
slower than the DDR bus connecting DRAM to the CPU. Consequently, tapping the
SPI bus is much cheaper than tapping the DDR bus. For this reason, systems
whose security relies on a cryptographiic hash of the firmware will first copy
the firmware into DRAM, hash the DRAM copy of the firmware, and then execute
the firmware from DRAM.

Although the DDR bus's speed makes tapping it very difficult, there are
well-publicized records of successful attempts. The original Xbox console's
booting process was reverse-engineered thanks to a passive tap on the DRAM
bus~\cite{huang2003xbox} that showed that the firmware used to boot the
console was partially stored in its southbridge. The protection mechanisms of
the PlayStation 3 hypervisor were subverted by an active tap on its memory
bus~\cite{hotz2010ps3} that targeted the hypervisor's page tables.


\subsubsection{Chip Tampering Attacks}


The CPU's chip
package is generally considered to be a minimal trust boundary, because
physical attacks that can breach the CPU modify
The most
rigurous threat models only assume that


\subsubsection{Power Analysis Attacks}
