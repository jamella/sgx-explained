\subsection{Physical Attacks}
\label{sec:physical_attacks}

Physical attacks are generally classified according to their cost, which
factors in the equipment needed to carry out the attack and the attack's
complexity. Joe Grand's DefCon presentation~\cite{grand2004physicalattacks}
provides a good overview with a large number of intuition-building figures and
photos.

The simplest type of physical attack is a denial of service attack performed by
disconnecting the victim computer's power supply or network cable. The threat
models of most secure architectures ignore this attack, because denial of
service can also be achieved by software attacks that compromise system
software such as the hypervisor.


\subsubsection{Port Attacks}
\label{sec:physical_port_attacks}

Slightly more involved attacks rely on connecting a device to an existing port
on the victim computer's case or motherboard~(\S~\ref{sec:motherboard}). A
simple example is a \textit{cold boot attack}, where the attacker plugs in a
USB flash drive into the victim's case and causes the computer to boot from
the flash drive, whose malicious system software receives unrestricted access
to the computer's peripherals.

More expensive physical attacks that still require relatively little effort
target the debug ports of various peripherals. The cost of these attacks is
generally dominated by the expense of acquiring the development kits needed to
connect to the debug ports. For example, recent Intel processors include the
Generic Debug eXternal Connection~(GDXC)~\cite{yuffe2011sandybridge,
intel2011gdxc}, which collects and filters the data transfered by the uncore's
ring bus (\S~\ref{sec:cache_coherence}), and reports it to an external
debugger.

The threat models of secure architectures generally ignore debug port attacks,
under the assumption that devices sold for general consumption have their debug
ports irreversibly disabled. In practice, manufacturers have strong incentives
to preserve debugging ports in production hardware, as this facilitates the
diagnosis and repair of defective units. Due to insufficient documentation on
this topic, we ignore the possibility of GDXC-based attacks. Fortunately, a
recent Intel patent~\cite{shanbhogue2015gdxcsgx} indicates that Intel engineers
are tackling at least some classes of attacks targeting debugging ports.


\subsubsection{Bus Tapping Attacks}
\label{sec:physical_bus_attacks}

More complex physical attacks consist of installing a device that taps a bus on
the computer's motherboard (\S~\ref{sec:motherboard}). \textit{Passive attacks}
are limited to monitoring the bus traffic, whereas \textit{active attacks} can
modify the traffic, or even place new commands on the bus. \textit{Replay
attacks} are a notoriously difficult to defeat class of active attacks, where
the attacker first records the bus traffic, and then selectively replays a
subset of the traffic. Replay attacks bypass systems that rely on static
signatures or HMACs, and generally aim to double-spend a limited resource.

The cost of bus tapping attacks is generally dominated by the cost of the
equipment used to tap the bus, which increases with bus speed and complexity.
For example, the flash chip that stores the computer's firmware is connected to
the PCH via an SPI bus (\S~\ref{sec:motherboard}), which is simpler and much
slower than the DDR bus connecting DRAM to the CPU. Consequently, tapping the
SPI bus is much cheaper than tapping the DDR bus. For this reason, systems
whose security relies on a cryptographic hash of the firmware will first copy
the firmware into DRAM, hash the DRAM copy of the firmware, and then execute
the firmware from DRAM.

Although the speed of the DDR bus makes tapping very difficult, there are
well-publicized records of successful attempts. The original Xbox console's
booting process was reverse-engineered thanks to a passive tap on the DRAM
bus~\cite{huang2003xbox}, which showed that the firmware used to boot the
console was partially stored in its southbridge. The protection mechanisms of
the PlayStation 3 hypervisor were subverted by an active tap on its memory
bus~\cite{hotz2010ps3} that targeted the hypervisor's page tables.

\S~\ref{sec:related} shows that concealing the addresses of the DRAM cells
accessed by a program is orders of magnitude more expensive than protecting the
memory's contents. Therefore, we are interested in analyzing attacks that tap
the DRAM bus, but only use the information on the address lines. These attacks
use the same equipment as normal DRAM bus tapping attacks, but require a
significantly more involved analysis to learn useful information. One of the
difficulties of such attacks is that the memory addresses observed on the DRAM
bus are generally very different from the application's memory access patterns,
because of the extensive cache hierachies in modern processors
(\S~\ref{sec:caching}).

We are not aware of any successful attack based on tapping the address lines of
a DRAM bus and analyzing the sequence of memory addresses.


\subsubsection{Chip Attacks}
\label{sec:physical_chip_attacks}

The most equipment-intensive physical attacks involve removing a chip's
packaging and directly interacting with its electrical circuits. These attacks
generally take advantage of equipment and techiques that were originally
developed to diagnose design and manufacturing defects in chips.
\cite{beck1998integrated} covers these techniques in depth.

The cost of chip attacks is dominated by the required equipment, although the
reverse-engineering involved is also non-trivial. This cost grows very rapidly
as the size of the circuitry's components decreases. At the time of this
writing, the latest Intel CPUs have a 14nm feature size, which requires ion
beam microscopy (TODO: look up the details for all this).

The least expensive class of chip attacks are destructive, and only require
imaging the chip's circuitry. These attacks rely on a microscope capable of
capturing the necessary details in each layer, and equipment for mechanically
removing each layer and exposing the layer below it to the microscope.

Imaging attacks generally target global secrets shared by all the chips in a
family, such as ROM masks that store global encryption keys or secret boot
code. They are also used to reverse-engineer undocumented functionality, such
as debugging backdoors. E-fuses and polyfuses are particularly vulnerable to
imaging attacks, because of their relatively large sizes.

Non-destructive passive chip attacks require measuring the voltages across a
module at specific times, while the chip is operating. These attacks are orders
of magnitude more expensive than imaging attacks, because the attacker must
maintain the integrity of the chip's circuitry, and therefore cannot de-layer
the chip.

The simplest active attacks on a chip consist in creating or destroying an
electric connection between two components. For example, the debugging
functionality in many chips is disabled by ``blowing'' an e-fuse. Once this
e-fuse is located, an attacker can reconnect its two ends, effectively undoing
the ``blowing'' operation. More expensive attacks involve changing voltages
across a component as the chip is operating, and are typically used to
reverse-engineer complex circuits.

Surprisingly, active attacks are not significantly more expensive to carry out
than passive non-destructive attacks. This is because the tools used to measure
the voltage across specific components are not very different from the tools
that can tamper with the chip's electric circuits. Therefore, once an attacker
develops a process for accessing a module without destroying the chip's
circuitry, the attacker can use the same process for both passive and active
attacks.

At the architectural level, we cannot address physical attacks against the
CPU's chip package. Active attacks on the CPU change the computer's execution
semantics, leaving us without any hardware that can be trusted to make security
decisions. Passive attacks can read the private data that the CPU is
processing. Therefore, many secure computing architectures assume that the
processor chip package is invulnerable to physical attacks.

Thankfully, physical attacks can be deterred by reducing the value that an
attacker obtains by compromising an individual chip. As long as this value is
below the cost of carrying out the physical attack, a system's designer can
hope that the processor's chip package will not be targeted by the physical
attacks.

Architects can reduce the value of compromising an individual system by
avoiding shared secrets, such as global encryption keys. Chip designers can
increase the cost of a physical attack by not storing a platform's secrets in
hardware that is vulnerable to destructive attacks, such as e-fuses.


\subsubsection{Power Analysis Attacks}
\label{sec:power_analysis_attacks}

An entirely different approach to physical attacks consists of indirectly
measuring the power consumption of a computer system or its components. The
attacker takes advantage of a known correlation between power consumption and
the data being computed on, and learns some property of the data from the
observed power consumption.

The earliest power analysis attacks have directly measured the processor chip's
power consumption. For example, [] used the correlation between the power
consumed by a smart card chip's multiplication circuitry and the number of one
bits in the input, and learned an RSA key that the smart card was supposed to
safeguard.

While direct power analysis attacks necessitate some equipment, their costs are
dominated by the complexity of the analysis required to learn the desired
information from the observed power trace which, in turn, is determined by the
complexity of the processor's circuitry. Today's smart cards contain special
circuitry designed to frustrate power analysis attacks. [] We are not aware of
any successful power analysis attacks against full-blown out-of-order Intel
processors.

Unfortunately, power analysis attacks can be extended to displays and human
input devices, which cannot be secured in any reasonable manner. For example,
[] measured the radiation emitted by a CRT display's ion beam to reconstitute
the image on a computer screen in a different room. [] used a directional
microphone to measure the sound emitted by a keyboard and learn the password
that its operator typed. [] applied similar techniques to learn a user's input
on a smartphone's on-screen keyboard, based on data from the device's
accelerometer.

In general, power attacks cannot be addressed at the architectural level, as
they rely on implementation details that are decided during the manufacturing
process. Therefore, it is unsurprising that the secure computing architectures
described in \S~\ref{sec:related} do not protect against power analysis
attacks.
