\subsection{Cache Coherence}
\label{sec:cache_coherence}

The Intel architecture was designed to support application software that was
not written with caches in mind. One aspect of this is the Total Store Order
(TSO) \cite{owens2009tso} memory model, the guarantee that concurrently running
hardware threads see the same order of DRAM writes. Given that a memory
location might be copied in multiple caches on different cores, or even
different processors, providing the TSO guarantees requires a \textit{cache
coherence protocol} that keeps the cache line copies in sync. This section
covers some cache coherence implementation details that are necessary for
understanding SGX. \cite{hennessy2012architecture} provides a good introduction
to cache coherence principles.

The cache coherence mechanism is not visible to software, so it only briefly
mentioned in the SDM. Fortunately, Intel's optimization reference
\cite{intel2014optimization} and the datasheets referenced in
\S~\ref{sec:cpu_die} provide more information. Intel processors use variations
of the MESIF \cite{goodman2009mesif} protocol, which is implemented in the CPU
and in the protocol layer of the QPI bus.

The SDM and the CPUID instructions indicate that the L3 cache, also known as
the \textit{last-level cache} (LLC) is \textit{inclusive}, meaning that any
location cached by an L1 or L2 cache must also be cached in the LLC. This
design decision reduces complexity in many implementation aspects. We estimate
that the bulk of the cache coherence implementation is in the CPU's uncore,
because cache synchronization can be achieved without having to communicate to
the lower cache levels, which are inside execution cores.

Unfortunately, a cache timing attack can take advantage of the fact that the
LLC is inclusive and shared among CPU cores. This allows an attacker thread
to monitor a target thread that runs on a core in the same CPU die. The
attacker can evict lines in the target core's cache by filling up the L3 cache,
and then probe the L3 cache to find out when the target causes cache evictions.
The evicted lines disclose some bits in the target thread's memory accesses.

The QPI protocol defines \textit{cache agents}, which are connected to the
last-level cache in a processor, and \textit{home agents}, which are connected
to memory controllers. Cache agents make requests to home agents for cache line
data on cache misses, while home agents keep track of cache line ownership, and
obtain the cache line data from other cache line agents, or from the memory
controller. The QPI routing layer supports multiple agents per socket, and each
processor has its own caching agents, and at least one home agent.

The CPU uncore (see Figure~\ref{fig:cpu_die}) has a bidirectional ring
interconnect used for communication between execution cores and the other
uncore components. The execution cores are connected to the ring by
\textit{CBoxes}, which route their LLC accesses. The routing is static, as the
LLC is divided into same-size slices (common slice sizes are 1.5Mb and 2.5Mb),
and an undocumented hashing scheme maps each possible physical address to
exactly one LLC slice. Intel's documentation states that the hashing scheme was
designed to avoid having a slice become a hotspot. The hashing scheme is the
reason why the L3 cache is documented as having a ``complex'' indexing scheme,
as opposed to the direct indexing used in the L1 and L2 caches.

The number of LLC slices matches the number of cores in the CPU, and each LLC
slice shares a CBox with a core. The CBoxes implement the cache coherence
engine, so each CBox acts as the QPI cache agent for its LLC slice. CBoxes
use a \textit{Source Address Decoder} (SAD) to route DRAM requests to the
appropriate home agents. Conceptually, the SAD takes in a memory address and
access type, and outputs a transaction type (coherent, non-coherent, IO) and a
node ID. Each CBox contains a SAD replica, and the configurations of all SADs
in a package are identical.

The SAD configurations are kept in sync by the \textit{UBox}, which is the
uncore configuration controller, and connects the \textit{System agent} to the
ring. The UBox is responsible for reading and writing physically distributed
registers across the uncore. The UBox also receives interrupts from system and
dispatches them to the appropriate core.

On recent Intel processors, the uncore also contains at least one memory
controller. Each integrated memory controller (iMC or MBox in Intel's
documentation) is connected to the ring by a \textit{home agent} (HA or
\textit{BBox} in Intel's datasheets). Each home agent contains a
\textit{Target Address Decoder} (TAD), which maps each physical DRAM address to
a specific DRAM channel.
