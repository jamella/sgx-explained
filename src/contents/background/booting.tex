\subsection{The Boot Process}
\label{sec:booting}

When a computer is powered up, it undergoes a \textit{bootstrapping} process,
also called \textit{booting}, for simplicity. Although many steps in the boot
process depend on the motherboard and components in a computer, the process
does follow a high-level structure that is prescribed in the SDM. This section
provides the details needed to analyze SGX's security properties.
\cite{intel2010booting} provides a good reference on the entire booting
process.

% Initialization Overview: SDM S 9.1

Right after a computer is powered up, all the logical processors (LPs) on the
motherboard undergo \textit{hardware initialization}, which invalidates the
caches (\S~\ref{sec:caching}) and TLBs (\S~\ref{sec:tlbs}), performs a
\textit{Built-In Self Test} (BIST), and sets all the registers
(\S~\ref{sec:registers}) to pre-specified values.

% Multiple-Processor Initialization: SDM S 8.4
% BSP and AP Processors: SDM S 8.4.1
% MP Initialization Protocol Algorithms for MP Systems: SDM S 8.4.3
% An ivy bridge CPUID: family 06h, extended model 3, model 58, stepping 9

After hardware initialization, the LPs perform the Multi-Processor (MP)
initialization algorithm, which results in one LP being selected as the
\textit{bootstrap processor} (BSP), and all the other LPs being classified as
\textit{application processors} (APs).

According to the SDM, the details of the MP initialization algorithm for recent
CPUs depend on the motherboard and firmware. In principle, after completing
hardware initialization, all LPs attempt to issue a special no-op transaction
on the QPI bus. A single LP will suceed in issuing the no-op, thanks to
the QPI arbitration mechanism, and to the UBox (\S~\ref{sec:cache_coherence})
in each CPU package, which also serves as a ring arbiter. The arbitration
priority of each LP is based on its APIC ID APIC ID (\S~\ref{sec:interrupts}),
which is provided by the motherboard when the system powers up. The LP that
issues the no-op becomes the BSP. Upon failing to issue the no-op, the other
LPs become APs, and enter the \textit{wait-for-SIPI} state.

% Typical BSP Initialization Sequence: SDM S 8.4.4.1

The BSP sets its RIP register to point to the firmware reset code, which must
be present at 0xFFFFFFF0 (16 bytes below the 4 GB mark). This is accomplished
by having the initial SAD (\S~\ref{sec:cache_coherence}) and PCH
(\S~\ref{sec:motherboard}) configurations map the 4 KB below the 4 GB mark of
the memory address space (\S~\ref{sec:address_spaces}) to the SPI flash chip
that stores the motherboard's firmware.

\cite{intel2010booting} and \cite{coreboot2015manual} describe the
initialization steps performed by the firmware, from an implementor's
perspective. A few steps are interesting from the perspective of SGX and
caching attacks.

% Preventing Caching: SDM S 11.5.3

When the BSP starts executing firmware code, DRAM is not available. The
firmware places the BSP in \textit{Cache-as-RAM} (CAR) mode to be able to use a
call stack and other high-level constructs. Ater CAR is enabled, the memory
initialization code, which is typically Intel's \textit{Memory Reference Code}
(MRC), is loaded into the cache. When executed, the memory initialization code
discovers the DRAM chips connected to the motherboard and sets them up, and
enables and configures the memory controllers.

% Typical AP Initialization Sequence: SDM S 8.4.4.2

