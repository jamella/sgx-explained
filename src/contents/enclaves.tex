\section{Memory Layout}
\label{sec:enclaves}

The central context of SGX is the \textit{enclave}, a protected environment
that contains the code and data pertaining to a security-sensitive computation.
This section provides an overview of the data structures used by an enclave.


\subsection{Enclaves and the RAM}

% PRM: SGX S 3.5

The enclaves' code and data is stored in \textit{Processor Reserved Memory}
(PRM), a contiguous range of RAM that cannot be directly accessed by other
software, including privileged software such as the SMM code, the hypervisor,
and the OS kernel. The memory controller is integrated on the CPU die (see
Figure~\ref{fig:cpu_die}), so it can be trusted to prevent devices attached to
the system bus from performing DMA transfers to/from the PRM.

Rejecting improper accesses to PRM is central to the SGX security model. The
designers were aware of that, and took clear steps to reduce the complexity of
the PRM checks, which in turn reduces the probability of bugs in the
implementation. The PRM must be set up to use SGX features and, once
configured, the PRM range cannot be changed. Furthermore, the PRM's size must
be an integer power of two, and its start address must be aligned to the same
power of two. The range restrictions reduce the complexity of checking if a RAM
address is in the PRM to a bitwise AND and an equality comparison.

% EPC and EPCM: SGX S 1.5, S 1.5.1, S 2.6.13, S 3.5, S 3.5.1

The contents of enclaves and the associated data structures are stored in the
\textit{Enclave Page Cache} (EPC). The EPC is a subset of the PRM, so the
protection measures described in the paragraphs above ensure that the enclaves'
memory cannot be read or tampered with by any malicious software running on the
host computer, or by malicious peripherals attached to the system bus. The
stringent restrictions placed on PRM documented above reduce the probability of
bugs in the security checks at the foundation for SGX's integrity and privacy
guarantees.

The EPC is split into 4kb pages, which are allocated to enclaves or supporting
data structures by the \textit{system software}, which can be either a
\textit{hypervisor} (the software running in VMX root mode at ring 0), or a
\textit{kernel} (the code inside an operating system running at ring 0).

The CPU maintains some metadata for each EPC page into the \textit{Enclave Page
Cache Map} (EPCM), which is used to ensure that an enclave does not attempt to
access another enclave's pages, and that system software manages EPC pages in a
way that is consistent with the SGX security model. The SGX documentation does
not state where the EPCM is stored, but we can hypothesize that it is either
an on-chip memory, like the L3 cache, or stored in a PRM region that is not
used by the EPC.

% SECINFO: SGX S 2.6.5, S 2.6.5.{1,2}


\subsection{Enclave Structures}

% SECS: SGX S 2.6.1, S 2.6.1.1, S 3.1

% ECREATE: SGX S 3.1, S 5.3
The first step in creating an enclave is using the \texttt{ECREATE} instruction
to designate an EPC page that becomes the enclave's \textit{SGX Enclave Control
Structure} (SECS). All other instructions that operate on an enclave access the
enclave's SECS directly or indirectly.



The EPCM keeps track of which pages contain SECS





% TCS: SGX S 2.6.2, S 2.6.2.{1,2,3,4}


% SSA: SGX S 2.6.3,
