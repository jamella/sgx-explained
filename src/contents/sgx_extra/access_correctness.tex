\subsection{SGX Security Check Correctness}
\label{sc:sgx_access_correctness}

SGX's security revolves around maintaining the following
invariant. \textbf{At all times, a logical processor's TLB
entries can only map DRAM pages that can be accessed by the code executing on
the processor.} Specifically, when a logical processor is in enclave mode, its
TLB can include entries for the enclave's EPC pages, and for DRAM pages outside
the PRM. The TLB of a logical processor outside enclave mode must not include
any PRM entry.

Without any special measures, the invariant described above would be broken
when a logical processor exits an enclave, either via
\texttt{EEXIT}~(\S~\ref{sec:sgx_eexit}), or via an AEX~(\S~\ref{sec:sgx_aex}).
This is because enclave mode allows TLB entries that point to the currently
executing enclave's EPC pages, and these entries become disallowed the moment
the processor leaves enclave mode. The SGX implementation solves this problem
by flushing a logical processor's TLBs when it leaves enclave mode.


% Enclave cannot read own SECS, because SECS contains enclave key
%   US 8,972,746 B2 - 9:21-25, 9:51
% Enclave cannot modify TCS
%   US 8,972,746 B2 - 10:1-6


