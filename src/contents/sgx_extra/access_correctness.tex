\subsection{SGX Security Check Correctness}
\label{sc:sgx_access_correctness}

In \S~\ref{sec:sgx_access_concepts}, we argued that SGX's security guarantees
can be obtained by modifying the Page Miss Handler~(PMH,~\S~\ref{sec:tlbs}) to
block undesirable address translations from reaching the TLB. This section
builds on the result above and  outlines a correctness proof for SGX's memory
access protection.

Specifically, we outline a proof for the following invariant. \textbf{At all
times, all the TLB entries in every logical processor will be consistent with
SGX's security guarantees.} By the argument in
\S~\ref{sec:sgx_access_concepts}, the invariant translates into an assurance
that all the memory accesses performed by software obey SGX's security model.
The high-level proof structure is presented because it helps understand how the
SGX security checks come together. By contrast, a a detailed proof would be
incredibly tedious, and would do very little to boost the reader's
understanding of SGX.


\subsubsection{Top-Level Invariant Breakdown}

We first break down the above invariant into specific cases based on whether
a logical processor~(LP) is executing enclave code or not, and on whether the
TLB entries translate virtual addresses in the current enclave's
ELRANGE~(\S~\ref{sec:sgx_elrange}). When the processor is outside enclave mode,
ELRANGE can be considered to be empty. This reasoning yields the three cases
outlined below.

\begin{enumerate}
\item At all times when an LP is outside enclave mode, its TLB may only contain
  physical addresses belonging to DRAM pages outside the PRM.
\item At all times when an LP is inside enclave mode, the TLB entries for
  virtual addresses outside the current enclave's ELRANGE must contain physical
  addresses belonging to DRAM pages outside the PRM.
\item At all times when an LP is in enclave mode, the TLB entries for virtual
  addresses inside the current enclave's ELRANGE must match the virtual memory
  layout specified by the enclave author.
\end{enumerate}

The first two invariant cases can be easily proven independently for each LP,
by induction over the sequence of instructions executed by the LP. For
simplicity, the reader can assume that instructions are executed in program
mode. While the assumption is not true on processors with out-of-order
execution~(\S~\ref{sec:out_of_order}), the arguments presented here also hold
when the executed instruction sequence is considered in retirement order, for
reasons that will be described below.

% Invalidating the Translation Lookaside Buffers (TLBs): SDM S 11.9
% MTRR Considerations in MP Systems: SDM S 11.11.8
% Loading MSRs: SDM S 26.4, S 26.7
% Implicit Accesses: SDM S 38.5.3.2
% Synchronous Entry and Exit: SDM S 39.2.1
% EENTER - Enters an Enclave: SDM S 41.4.1
% EEXIT - Exits an Enclave: SDM S 41.4.1
% ERESUME - Re-Enters an Enclave: SDM S 41.4.1
% AEX Operational Detail: SDM S 40.4.1
% Interactions with Caching of Linear-Address Translations: SDM S 42.13

An LP will only transition between enclave mode and non-enclave mode at a few
well-defined points, which are \texttt{EENTER}~(\S~\ref{sec:sgx_eenter}),
\texttt{ERESUME}~(\S~\ref{sec:sgx_eresume}),
\texttt{EEXIT}~(\S~\ref{sec:sgx_eexit}),
and Asynchronous Enclave Exits~(AEX,~\S~\ref{sec:sgx_aex}). According to the
SDM, all the transition points flush the TLBs and the out-of-order execution
pipeline. In other words, the TLBs are guaranteed to be empty after every
transition between enclave mode and non-enclave mode, so we can consider all
these transitions to be trivial base cases for our induction proofs.

While SGX initialization is not thoroughly discussed, the SDM mentions that
loading some Model-Specific Registers~(MSRs,~\S~\ref{sec:address_spaces})
triggers TLB flushes, and that system software should flush TLBs when modifying
Memory Type Range Registers~(MTRRs,~\S~\ref{sec:cacheability_config}). Given
that all the possible SGX implementations described in
\S~\ref{sec:sgx_pmh_hardware} entail adding a MTRR, it is safe to assume that
enabling SGX mode also results in a TLB flush and out-of-order pipeline flush,
and can be used by our induction proof as well.

All the base cases in the induction proofs are serialization points for
out-of-order execution, as the pipeline is flushed during both enclave mode
transitions and SGX initialization. This makes the proofs below hold when the
program order instruction sequence is replaced with the retirement order
sequence.

The first invariant case holds because while the LP is outside enclave mode,
the SGX security checks added to the PMH (\S~\ref{sec:sgx_access_concepts},
Figure~\ref{fig:sgx_tlb_miss_checks}) reject any address translation that would
point into the PRM before it reaches the TLBs. A key observation for proving
the induction step of this invariant case is that the PRM never changes after
SGX is enabled on an LP.

The second invariant case can be proved using a similar argument. While an LP
is executing an enclave's code, the SGX memory access checks added to the PMH
reject any address translation that resolves to a physical address inside the
PRM, if the translated virtual address falls outside the current enclave's
ELRANGE. The induction step for this invariant case can be proven by observing
that a change in an LP's current ELRANGE is always accompanied by a TLB flush,
which results in an empty TLB that trivially satisfies the invariant. This
follows from the constraint that an enclave's ELRANGE never changes after it is
established, and from the observation that the LP's current enclave can only be
changed by an enclave entry, which must be preceded by an enclave exit, which
triggers a TLB flush.

The third invariant case is best handled by recognizing that the Enclave Page
Cache Map~(EPCM,~\S~\ref{sec:sgx_epcm}) is an intermediate representation for
the virtual memory layout specified by the enclave authors. This suggests
breaking down the case into smaller sub-invariants centered around the EPCM,
which will be proven in the sub-sections below.

\begin{enumerate}
\item At all times, each EPCM entry for a page that is allocated to an enclave
  matches the virtual memory layout desired by the enclave's author.
\item Assuming that the EPCM contents is constant, at all times when an LP is
  in enclave mode, the TLB entries for virtual addresses inside the current
  enclave's ELRANGE must match EPCM entries that belong to the enclave.
\item An EPCM entry is only modified when there is no mapping for it in any
  LP's TLB.
\end{enumerate}

The second and third invariant combined prove that all the TLBs in an
SGX-enabled computer always reflect the contents of the EPCM, as the third
invariant essentially covers the gaps in the second invariant. This result, in
combination with the first invariant, shows that the EPCM is a bridge between
the memory layout specifications of the enclave authors and the TLB entries
that regulate what memory can be accessed by software executing on the LPs.
When further combined with the reasoning in \S~\ref{sec:sgx_access_concepts},
the whole proof outlined here results in an end-to-end argument for the
correctness of SGX's memory protection scheme.


\subsubsection{EPCM Entries Reflect Enclave Author Design}

This sub-section outlines the proof for the following invariant. \textbf{At
all times, each EPCM entry for a page that is allocated to an enclave matches
the virtual memory layout desired by the enclave's author.}

A key observation, backed by the SDM pseudocode for SGX instructions, is that
all the instructions that modify the EPCM pages allocated to an enclave are
synchronized using a lock in the enclave's SECS. This entails the existence of
a time ordering of the EPCM modifications associated with an enclave. We prove
the invariant stated above using a proof by induction over this sequence of
EPCM modifications.

EPCM entries allocated to an enclave are created by instructions that can only
be issued before the enclave is initialized, specifically
\texttt{ECREATE}~(\S~\ref{sec:sgx_ecreate}) and
\texttt{EADD}~(\S~\ref{sec:sgx_eadd}). The contents of the EPCM entries created
by these instructions contributes to the enclave's
measurement~(\S~\ref{sec:sgx_measurement}), together with the initial data
loaded into the corresponding EPC pages.

\S~\ref{sec:generic_measurement} argues that we can assume that enclaves with
incorrect measurements do not exist, as they will be rejected by software
attestation. Therefore, we can assume that the attributes used to initialize
EPCM pages match the enclave authors' memory layout specifications.

EPCM entries can be evicted to untrusted DRAM, together with their
corresponding EPC pages, by the \texttt{EWB}~(\S~\ref{sec:sgx_ewb})
instruction.  The \texttt{ELDU} / \texttt{ELDB}~(\S~\ref{sec:sgx_epc_eviction})
instructions re-load evicted page contents and metadata back into the EPC and
EPCM. By induction, we can assume that an EPCM entry matches the enclave
author's specification when it is evicted. Therefore, if we can prove that the
EPCM entry that is reloaded from DRAM is equivalent to the entry that was
evicted, we can conclude that the reloaded enrtry matches the author's
specification.

A detailed analysis of the cryptographic primitives used by the SGX design to
protect the evicted EPC page contents and its associated metadata is outside
the scope of this work. Summarizing the description in
\S~\ref{sec:sgx_epc_eviction}, the contents of evicted pages is encrypted using
AES-GMAC~(\S~\ref{sec:integrity_crypto}), which is an authenticated encryption
mechanism. The MAC tag produced by AES-GMAC covers the EPCM metadata as well as
the page data, and includes a 64-bit version that is stored in a version tree
whose nodes are Version Array~(VA,~(\S~\ref{sec:sgx_va}) pages.

Assuming no cryptographic weaknesses, SGX's scheme does appear to guarantee the
privacy, integrity, and freshness of the EPC page contents and associated
metadata while it is evicted in untrusted memory. It follows that \texttt{EWB}
will only reload an EPCM entry if the contents is equivalent to the contents of
an evicted entry.

The equivalence notion invoked here is slightly different from perfect
equality, in order to account for the allowable operation of evicting an EPC
page and its associated EPCM entry, and then reloading the page contents to a
different EPC page and a different EPCM entry, as illustrated in
Figure~\ref{fig:sgx_page_eviction}. Loading the contents of an EPC page at a
different physical address than it had before does not break the virtual memory
abstraction, as long as the contents is mapped at the same virtual address
(the LINEARADDRESS EPCM field), and has the same access control attributes
(R, W, X, PT EPCM fields) as it had when it was evicted.


\subsubsection{TLB Entries for ELRANGE Reflect EPCM Contents}

This sub-section sketches a proof for the following invariant. \textbf{At all
times when an LP is in enclave mode, the TLB entries for virtual addresses
inside the current  enclave's ELRANGE must match EPCM entries that belong to
the enclave.} The argument makes the assumption that \textbf{the EPCM contents
is constant}, which will be justified in the following sub-section.

The invariant can be proven by induction over the sequence of TLB insertions
that occur in the LP. This sequence is well-defined because an LP has a single
PMH, so the address translation requests triggered by TLB misses must be
serialized to be processed by the PMH.

The proof's induction step depends on the fact that the TLB on hyper-threaded
cores~(\S~\ref{sec:cpu_core}) is dynamically partitioned between the two LPs
that share the core, and no TLB entry is shared between the LPs. This allows
our proof to consider the TLB insertions associated with one LP independently
from the other LP's insertions, which means we don't have to worry about the
state (e.g., enclave mode) of the other LP on the core.

The proof is further simplified by observing that when an LP exits enclave
mode, both its TLB and its out-of-order instruction pipeline are flushed.
Therefore, the enclave mode and current enclave register values used by address
translations are guaranteed to match the values obtained by performing the
translations in program order.

Having eliminated all the complexities associated with
hyper-threaded~(\S~\ref{sec:cpu_core}) out-of-order~(\S~\ref{sec:out_of_order})
execution cores, it is easy to see that the security checks outlined in
Figure~\ref{fig:sgx_tlb_miss_checks} and \S~\ref{sec:sgx_access_concepts}
ensure that TLB entries that target EPC pages are guaranteed to reflect the
constraints in the corresponding EPCM entries.

Last, the SGX access checks implemented in the PMH reject any address
translation for a virtual address in ELRANGE that does not resolve to an EPC
page. It follows that memory addresses inside ELRANGE can only map to EPC
pages which, by the argument above, must follow the constraints of the
corresponding EPCM entries.


\subsubsection{EPCM Entries are Not In TLBs When Modified}

In this sub-section, we outline a proof that \textbf{an EPCM entry is only
modified when there is no mapping for it in any LP's TLB.}. This proof analyzes
each of the instructions that modify EPCM entries.

For the purposes of this proof, we consider that setting the BLOCKED attribute
does not count as a modification to an EPCM entry, as it does not change the
EPC page that the entry is associated with, or the memory layout specification
associated with the page.

The instructions that modify EPCM entries in such a way that the resulting EPCM
entries have the VALID field set to true require that the EPCM entries were
invalid before they were modified. These instructions are
\texttt{ECREATE}~(\S~\ref{sec:sgx_ecreate}),
\texttt{EADD}~(\S~\ref{sec:sgx_eadd}), \texttt{EPA}~(\S~\ref{sec:sgx_epa}),
and \texttt{ELDU} / \texttt{ELDB}~(\S~\ref{sec:sgx_epc_eviction}). The EPCM
entry targeted by any these instructions must have had its VALID field set to
false, so the invariant proved in the previous sub-section implies that the
EPCM entry had no TLB entry associated with it.

Conversely, the instructions that modify EPCM entries and result in entries
whose VALID field is false start out with valid entries. These instructions are
\texttt{EREMOVE}~(\S~\ref{sec:sgx_eremove}) and
\texttt{EWB}~(\S~\ref{sec:sgx_ewb}).

The EPCM entries associated with EPC pages that store Version
Arrays~(VA,~\S~\ref{sec:sgx_va}) represent a special case for both instructions
mentioned above, as these pages are not associated with any enclave. As these
pages can only be accessed by the microcode used to implement SGX, they never
have TLB entries representing them. Therefore, both \texttt{EREMOVE} and
\texttt{EWB} can invalidate EPCM entries for VA pages without additional
checks.

\texttt{EREMOVE} only invalidates an EPCM entry associated with an enclave when
there is no LP executing in enclave mode using a TCS associated with the same
enclave. An EPCM entry can only result in TLB translations when an LP is
executing code from the entry's enclave, and the TLB translations are flushed
when the LP exits enclave mode. Therefore, when \texttt{EREMOVE} invalidates
an EPCM entry, any associated TLB entry is guaranteed to have been flushed.

\texttt{EWB}'s correctness argument is more complex, as it relies on the
\texttt{EBLOCK} / \texttt{ETRACK} sequence described in \S~\ref{sec:sgx_eblock}
to ensure that any TLB entry that might have been created for an EPCM entry is
flushed before the EPCM entry is invalidated.

Unfortunately, the SDM pseudocode for the instructions mentioned above leaves
out the algorithm used to verify that the relevant TLB entries have been
flushed. Thus, we must base our proof on the assumption that the SGX
implementation produced by Intel's engineers matches the claims in the SDM.
In \S~\ref{sec:sgx_ewb_guess}, we propose a method for ensuring that
\texttt{EWB} will only succeed when all the LPs executing an enclave's code at
the time when \texttt{ETRACK} is called have exited enclave mode at least once
between the \texttt{ETRACK} call and the \texttt{EWB} call. Having proven the
existence of a correct algorithm by construction, we can only hope that the SGX
implementation uses our algorithm, or a better algorithm that is still correct.

% Enclave cannot read own SECS, because SECS contains enclave key
%   US 8,972,746 B2 - 9:21-25, 9:51
% Enclave cannot modify TCS
%   US 8,972,746 B2 - 10:1-6
