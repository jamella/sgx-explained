\HeadingLevelB{Tracking TLB Flushes}
\label{sec:sgx_ewb_guess}

This section proposes a straightforward method that the SGX implementation can
use to verify that the system software plays its part correctly in the EPC page
eviction~(\S~\ref{sec:sgx_epc_eviction}) process. Our method meets the SDM's
specification for \texttt{EBLOCK}~(\S~\ref{sec:sgx_eblock}),
\texttt{ETRACK}~(\S~\ref{sec:sgx_eblock}) and
\texttt{EWB}~(\S~\ref{sec:sgx_ewb}).

The motivation behind this section is that, at least at the time of this
writing, there is no official SGX documentation that contains a description of
the mechanism used by \texttt{EWB} to ensure that all the Logical
Processors~(LPs,~\S~\ref{sec:cpu_core}) running an enclave's code exit enclave
mode~(\S~\ref{sec:sgx_threads}) between an \texttt{ETRACK} invocation and a
\texttt{EWB} invocation. Knowing that there exists a correct mechanism that has
the same interface as the SGX instructions described in the SDM gives us a
reason to hope that the SGX implementation is also correct.

Our method relies on the fact that an enclave's SECS~(\S~\ref{sec:sgx_secs}) is
not accessible by software, and is already used to store information used by
the SGX microcode implementation~(\S~\ref{sec:sgx_microcode_modifications}).
We store the following fields in the SECS. \id{tracking} and
\id{done-tracking} are Boolean variables. \id{tracked-threads} and
\id{active-threads} are non-negative integers that start at zero and must store
numbers up to the number of LPs in the computer. \id{lp-mask} is an array of
Boolean flags that has one member per LP in the computer. The fields are
initialized as shown in Figure~\ref{fig:sgx_tracking_ecreate}.

\begin{figure}[hbt]
  \begin{codebox}
  \Procname{$\proc{ECREATE}(\id{SECS})$}
  \zi \Comment Initialize the SECS state used for tracking.
  \li $\id{SECS}.\id{tracking} \gets \const{false}$
  \li $\id{SECS}.\id{done-tracking} \gets \const{false}$
  \li $\id{SECS}.\id{active-threads} \gets 0$
  \li $\id{SECS}.\id{tracked-threads} \gets 0$
  \li $\id{SECS}.\id{lp-mask} \gets 0$
  \end{codebox}
  \caption{
    The algorithm used to initialize the SECS fields used by the TLB flush
    tracking method presented in this section.
  }
  \label{fig:sgx_tracking_ecreate}
\end{figure}

The \id{active-threads} SECS field tracks the number of LPs that are currently
executing the code of the enclave who owns the SECS. The field is is atomically
incremented by \texttt{EENTER}~(\S~\ref{sec:sgx_eenter}) and
\texttt{ERESUME}~(\S~\ref{sec:sgx_eresume}) and is atomically decremented by
\texttt{EEXIT}~(\S~\ref{sec:sgx_eexit}) and Asynchronous Enclave
Exits~(AEXs,~\S~\ref{sec:sgx_aex}). Asides from helping track TLB flushes, this
field can also be used by \texttt{EREMOVE}~(\S~\ref{sec:sgx_eremove}) to decide
when it is safe to free an EPC page that belongs to an enclave.

As specified in the SDM, \texttt{ETRACK} activates TLB flush tracking for an
enclave. In our method, this is accomplished by setting the \id{tracking} field
to \const{true} and the \id{done-tracking} field to \const{false}.

When tracking is enabled, \id{tracked-threads} is the number of LPs that were
executing the enclave's code when the \texttt{ETRACK} instruction was issued,
and have not yet exited enclave mode. Therefore, executing \texttt{ETRACK}
atomically reads \id{active-threads} and writes the result into
\id{tracked-threads}. Also, \id{lp-mask} keeps track of the LPs that have
exited the current enclave after the \texttt{ETRACK} instruction was issued.
Therefore, the \texttt{ETRACK} implementation atomically zeroes \id{lp-mask}.
The full \texttt{ETRACK} algorithm is listed in
Figure~\ref{fig:sgx_tracking_etrack}.

\begin{figure}[hbt]
  \begin{codebox}
  \Procname{$\proc{ETRACK}(\id{SECS})$}
  \zi \Comment Abort if tracking is already active.
  \li \If $\id{SECS}.\id{tracking} = \const{true}$
  \li   \Then \Return $\const{sgx-prev-trk-incmpl}$
        \End
  \zi \Comment Activate TLB flush tracking.
  \li $\id{SECS}.\id{tracking} \gets \const{true}$
  \li $\id{SECS}.\id{done-tracking} \gets \const{false}$
  \li $\id{SECS}.\id{tracked-threads} \gets$
      \Indentmore
  \zi $\proc{atomic-read}(\id{SECS}.\id{active-threads})$
      \End
  \li \For $i \gets 0$ \To $\const{max-lp-id}$
  \li   \Do
          $\proc{atomic-clear}(\id{SECS}.\id{lp-mask}[i])$
        \End
  \end{codebox}
  \caption{
    The algorithm used by \texttt{ETRACK} to activate TLB flush tracking.
  }
  \label{fig:sgx_tracking_etrack}
\end{figure}

When an LP exits an enclave that has TLB flush tracking activated, we
atomically test and set the current LP's flag in \id{lp-mask}. If the flag was
not previously set, it means that an LP that was executing the enclave's code
when \texttt{ETRACK} was invoked just exited enclave mode for the first time,
and we atomically decrement \id{tracked-threads} to reflect this fact. In other
words, \id{lp-mask} prevents us from double-counting an LP when it exits the
same enclave while TLB flush tracking is active.

Once \id{active-threads} reaches zero, we are assured that all the LPs running
the enclave's code when \texttt{ETRACK} was issued have exited enclave mode at
least once, and can set the \id{done-tracking} flag.
Figure~\ref{fig:sgx_tracking_eexit} enumerates all the steps taken on enclave
exit.

\begin{figure}[hbt]
  \begin{codebox}
  \Procname{$\proc{enclave-exit}(\id{SECS})$}
  \zi \Comment Track an enclave exit.
  \li $\proc{atomic-decrement}(\id{SECS}.\id{active-threads})$
  \li \If $\proc{atomic-test-and-set}($
        \Indentmore
  \zi     $\id{SECS}.\id{lp-mask}[\const{lp-id}])$
        \End
  \li   \Then $\proc{atomic-decrement}($
          \Indentmore
  \zi     $\id{SECS}.\id{tracked-threads})$
          \End
  \li   \If $\id{SECS}.\id{tracked-threads} = 0$
  \li     \Then $\id{SECS}.\id{done-tracking} \gets \const{true}$
          \End
        \End
  \end{codebox}
  \caption{
    The algorithm that updates the TLB flush tracking state when an LP exits an
    enclave via \texttt{EEXIT} or AEX.
  }
  \label{fig:sgx_tracking_eexit}
\end{figure}

Without any compensating measure, the method above will incorrectly decrement
\id{tracked-threads}, if the LP exiting the enclave had entered it after
\texttt{ETRACK} was issued. We compensate for this with the following trick.
When an LP starts executing code inside an enclave that has TLB flush tracking
activated, we set its corresponding flag in \id{lp-mask}. This is sufficient to
avoid counting the LP when it exits the enclave.
Figure~\ref{fig:sgx_tracking_eenter} lists the steps required by our method
when an LP enters an enclave.

\begin{figure}[hbt]
  \begin{codebox}
  \Procname{$\proc{enclave-enter}(\id{SECS})$}
  \zi \Comment Track an enclave entry.
  \li $\proc{atomic-increment}(\id{SECS}.\id{active-threads})$
  \li $\proc{atomic-set}(\id{SECS}.\id{lp-mask}[\const{lp-id}])$
  \end{codebox}
  \caption{
    The algorithm that updates the TLB flush tracking state when an LP enters
    an enclave via \texttt{EENTER} or \texttt{ERESUME}.
  }
  \label{fig:sgx_tracking_eenter}
\end{figure}

With these algorithms in place, \texttt{EWB} can simply verify that both
\id{tracking} and \id{done-tracking} are \const{true}. This ensures that the
system software has triggered enclave exits on all the LPs that were running
the enclave's code when \texttt{ETRACK} was executed.
Figure~\ref{fig:sgx_tracking_ewb} lists the algorithm used by the \texttt{EWB}
tracking verification step.

\begin{figure}[hbt]
  \begin{codebox}
  \Procname{$\proc{EWB-verify}(\id{virtual-addr})$}
  \li $\id{physical-addr} \gets \proc{translate}(\id{virtual-addr})$
  \li $\id{epcm-slot} \gets \proc{epcm-slot}(\id{physical-addr})$
  \li \If $\id{EPCM}[\id{slot}].\id{BLOCKED} = \const{false}$
  \li   \Then \Return $\const{sgx-not-blocked}$
        \End
  \li $\id{SECS} \gets \proc{epcm-addr}($
      \Indentmore
  \zi   $\id{EPCM}[\id{slot}].\id{ENCLAVESECS})$
      \End
  \zi \Comment Verify that the EPC page can be evicted.
  \li \If $\id{SECS}.\id{tracking} = \const{false}$
  \li   \Then \Return $\const{sgx-not-tracked}$
        \End
  \li \If $\id{SECS}.\id{done-tracking} = \const{false}$
  \li   \Then \Return $\const{sgx-not-tracked}$
        \End
  \end{codebox}
  \caption{
    The algorithm that ensures that all LPs running an enclave's code when
    \texttt{ETRACK} was executed have exited enclave mode at least once.
  }
  \label{fig:sgx_tracking_ewb}
\end{figure}

Last, \texttt{EBLOCK} marks the end of a TLB flush tracking cycle by clearing
the \id{tracking} flag. This ensures that system software must go through
another cycle of \texttt{ETRACK} and enclave exits before being able to use
\texttt{EWB} on the page whose BLOCKED EPCM field was just set to \const{true}
by \texttt{EBLOCK}.
Figure~\ref{fig:sgx_tracking_eblock} shows the details.

\begin{figure}[hbt]
  \begin{codebox}
  \Procname{$\proc{EBLOCK}(\id{virtual-addr})$}
  \li $\id{physical-addr} \gets \proc{translate}(\id{virtual-addr})$
  \li $\id{epcm-slot} \gets \proc{epcm-slot}(\id{physical-addr})$
  \li \If $\id{EPCM}[\id{slot}].\id{BLOCKED} = \const{true}$
  \li   \Then \Return $\const{sgx-blkstate}$
        \End
  \li \If $\id{SECS}.\id{tracking} = \const{true}$
  \li   \Then
          \If $\id{SECS}.\id{done-tracking} = \const{false}$
  \li       \Then \Return $\const{sgx-entryepoch-locked}$
            \End
  \li     $\id{SECS}.\id{tracking} \gets \const{false}$
        \End
  \li $\id{EPCM}[\id{slot}].\id{BLOCKED} \gets \const{true}$
  \end{codebox}
  \caption{
    The algorithm that marks the end of a TLB flushing cycle when
    \texttt{EBLOCK} is executed.
  }
  \label{fig:sgx_tracking_eblock}
\end{figure}

Our method's correctness can be easily proven by arguing that each SECS field
introduced in this section has its intended value throughout enclave entries
and exits.
