\subsection{SGX Implementation Overview}
\label{sec:sgx_implementation_overview}

An under-documented and overlooked feat achieved by the SGX design is that
implementing it on an Intel processor has a very low impact on the chip's
hardware design. SGX's modifications to the processor's execution
cores~(\S~\ref{sec:cpu_core}) are either very small or completely inexistent.
The CPU's uncore~(\S~\ref{sec:cpu_die}, \S~\ref{sec:cache_coherence}) receives
a new module, the  Memory Encryption Engine, which appears to be fairly
self-contained.

The bulk of the SGX implementation is relegated to the processor's
microcode~(\S~\ref{sec:microcode}), which supports a much higher development
speed than the chip's electrical circuitry.


\subsubsection{Execution Core Modifications}
\label{sec:sgx_core_modifications}

The SGX design may require a very small modification to the processor's
execution cores~(\S~\ref{sec:cpu_core}), in the Page Miss
Handler~(PMH,~\S~\ref{sec:tlbs}). The PMH resolves TLB misses, and consists of
a fast path that relies on an FSM page walker, and a microcode assist fallback
that handles the edge cases (\S~\ref{sec:microcode_pmh}).

The bulk of SGX's memory access checks, which are discussed in
\S~\ref{sec:sgx_access_protection}, can be implemented in the microcode assist.
The only modification to the PMH hardware is having it trigger the microcode
assist when the physical address produced by the page walker FSM matches the
Processor Reserved Memory~(PRM,~\S~\ref{sec:sgx_prm}) range.

We state that SGX \textbf{may} require a hardware modification to the PMH
because there is a possibility that the behavior described above can be
achieved solely with microcode configuration, as explained below.

The PRM range is configured by the PRM Range Registers~(\S~\ref{sec:sgx_prm}),
which have exactly the same semantics as the Memory Type Range
Registers~(MTRRs,~\S~\ref{sec:cacheability_config}) used to configure a
variable memory range. The page walker FSM in the PMH is already configured to
issue a microcode assist when the page tables are in uncacheable
memory~(\S~\ref{sec:memory_io}). Therefore, the PRMRR can be represented as an
extra MTRR pair.


\subsubsection{Uncore Modifications}
\label{sec:sgx_uncore_modifications}

The SDM states that DMA transactions~(\S~\ref{sec:motherboard}) that target the
PRM range are aborted by the processor. The SGX patents disclose that the PRMRR
protection aginst unauthorized DMA is implemented by having the SGX microcode
set up entries in the Source Address Decoder~(SAD) in the uncore CBoxes and in
the Target Address Decoder~(TAD) in the integrated Memory Controller~(MC).

\S~\ref{sec:cache_coherence} mentions that Intel's Trusted Execution
Technology~(TXT)~\cite{grawrock2009txt} already takes advantage of the
integrated MC to protect a DRAM range from DMA. It is highly likely that the
SGX implementation reuses the mechanisms brought by TXT, and only requires the
extension of the SADs and TADs by one entry.

SGX's major hardware modification is the Memory Encryption Engine~(MEE) that is
added to the processor's uncore~(\S~\ref{sec:cpu_die},
\S~\ref{sec:cache_coherence}) to protect SGX's Enclave Page
Cache~(EPC,~\S~\ref{sec:sgx_epc}) against physical attacks.

% ISCA SGX Slides 163-201

The MEE is briefly described in the ISCA 2015 SGX
tutorial~\cite{intel2015iscasgx}. According to the information presented there,
the MEE roughly follows the approach introduced by Aegis \cite{suh2003aegis}
\cite{aegis_impl}, which relies on a variation of Merkle trees to provide the
EPC with privacy, integrity, and freshness
guarantees~(\S~\ref{sec:crypto_primitives}). Unlike Aegis, the MEE uses
non-standard cryptographic primitives that include a slightly modified AES
operating mode~(\S~\ref{sec:privacy_crypto}) and a
Carter-Wegman~\cite{carter1977mac, wegman1981mac}
MAC~(\S~\ref{sec:integrity_crypto}) construction.

% ISCA SGX Slide 167

Both the ISCA SGX tutorial and the patents state that the MEE is connected to
to the Memory Controller (MC) integrated in the CPU's uncore. However, all
sources are completely silent on further implementation details. The
MEE overview slide states that ``the Memory Controller detects [the] address
belongs to the MEE region, and routes transaction to MEE'', which suggests that
the MEE is fairly self-contained and has a narrow interface to the rest of the
MC.


\subsubsection{Microcode Modifications}
\label{sec:sgx_microcode_modifications}

According to the SGX patents, all the SGX instructions are implemented in
microcode. This can also be deduced by reading the SDM's pseuodocode for all
the instructions, and realizing that it is highly unlikely that any SGX
instruction can be implemented in 4 or fewer
micro-ops~(\S~\ref{sec:out_of_order}), which is the most that can be
handled by the simple decoders used in the hardware fast paths
(S~\ref{sec:microcode_role}).

The Asynchronous Enclave Exit~(AEX,~\S~\ref{sec:sgx_aex}) behavior is also
implemented in microcode. \S~\ref{sec:microcode_structure} draws on an
assortment of Intel patents to conclude that hardware
exceptions~(\S~\ref{sec:faults}), including both faults and interrupts,
trigger microcode events~(\S~\ref{sec:microcode_structure}). It follows that
the SGX implementation can implement AEX by modifying the hardware exception
handlers in the microcode.

% ISCA SGX Slide 166

Last, the SGX initialization sequence is also implemented in microcode. It is
likely that the boot sequence in microcode~(\S~\ref{sec:microcode_sec}) was
modified to initialize the registers associated with the SGX microcode. The
ISCA SGX tutorial states that the MEE's keys are initialized during the boot
process.

% Intel SGX Opt-In Configuration: SDM S 37.7.1

The SDM states that SGX instructions are enabled by setting a bit in a
Model-Specific Register~(MSR,~\S~\ref{sec:address_spaces}). Initializing SGX
requires enabling the MEE and configuring the SAD and TAD to protect the PRM
range, and both tasks are amenable to a microcode implementation.

The SGX patents state that SGX requires very few hardware changes, and most of
the implementation is in microcode, as a positive fact. We therefore suspect
that minimizing hardware changes was a high priority in the SGX design, and
that any SGX modification proposals need to be aware of this priority.
