\HeadingLevelB{Enclave Signature Verification}
\label{sec:sgx_rsa_check}

Let $m$ be the public modulus in the enclave author's RSA key, and $s$ be the
enclave signature. Since the SGX design fixes the value of the public exponent
$e$ to $3$, verifying the RSA signature amounts to computing the signed
message $M = s^3 \bmod m$, checking that the value meets the PKCS v1.5 padding
requirements, and comparing the 256-bit SHA-2 hash inside the message with the
value obtained by hashing the relevant fields in the SIGSTRUCT supplied with
the enclave.

This section describes an algorithm for computing the signed message while only
using subtraction and multiplication on large non-negative integers \cite{Gueron-Quick-Verification}. The
algorithm admits a significantly simpler implementation than the typical RSA
signature verification algorithm, by avoiding the use of long division and
negative numbers.

The algorithm provided here requires the signer to compute the $q_1$ and
$q_2$ values shown below. The values can be computed from the public
information in the signature, so they do not leak any additional information
about the private signing key. Furthermore, the algorithm verifies the
correctness of the values, so it does not open up the possibility for an attack
that relies on supplying incorrect values for $q_1$ and $q_2$.

\begin{align*}
q_1 & = \left\lfloor \frac{s^2}{m} \right\rfloor \\
q_2 & = \left\lfloor \frac{s^3 - q_1 \times s \times m}{m} \right\rfloor
\end{align*}

Due to the desirable properties mentioned above, it is very likely that the
algorithm described here is used by the SGX implementation to verify the RSA
signature in an enclave's SIGSTRUCT~(\S~\ref{sec:sgx_sigstruct}).

The algorithm in Figure~\ref{fig:sgx_sig_verification} computes the signed
message $M = s^3 \bmod m$, while also verifying that the given values of $q_1$
and $q_2$ are correct. The latter is necessary because the SGX implementation
of signature verification must handle the case where an attacker attempts to
exploit the signature verification implementation by supplying invalid values
for $q_1$ and $q_2$.

\begin{figure}[hbt]
  \begin{enumerate}
    \item Compute $u \leftarrow s \times s$ and $v \leftarrow q_1 \times m$
    \item If $u < v$, abort. $q_1$ must be incorrect.
    \item Compute $w \leftarrow u - v$
    \item If $w \ge m$, abort. $q_1$ must be incorrect.
    \item Compute $x \leftarrow w \times s$ and $y \leftarrow q_2 \times m$
    \item If $x < y$, abort. $q_2$ must be incorrect.
    \item Compute $z \leftarrow x - y$.
    \item If $z \ge m$, abort. $q_2$ must be incorrect.
    \item Output $z$.
  \end{enumerate}
  \caption{
    An RSA signature verification algorithm specialized for the case where
    the public exponent is 3. $s$ is the RSA signature and $m$ is the RSA key
    modulus. The algorithm uses two additional inputs, $q_1$ and $q_2$.
  }
  \label{fig:sgx_sig_verification}
\end{figure}

The rest of this section proves the correctness of the algorithm in
Figure~\ref{fig:sgx_sig_verification}.


\HeadingLevelC{Analysis of Steps 1 - 4}

Steps $1 - 4$ in the algorithm check the correctness of $q_1$ and use it
to compute $s^2 \bmod m$. The key observation to understanding these steps is
recognizing that $q_1$ is the quotient of the integer division $s^2 / m$.

Having made this observation, we can use elementary division properties to
prove that the supplied value for $q_1$ is correct if and only if the following
property holds.

$$ 0 \le s^2 - q_1 \times m < m $$

We observe that the first comparison, $0 \le s^2 - q_1 \times m$, is equivalent
to $q_1 \times m \le s^2$, which is precisely the check performed by step 2. We
can also see that the second comparison, $s^2 - q_1 \times m < m$ corresponds
to the condition verified by step 4. Therefore, if the algorithm passes step 4,
it must be the case that the value supplied for $q_1$ is correct.

We can also plug $s^2$, $q_1$ and $m$ into the integer division remainder
definition to obtain the identity $s^2 \bmod m = s^2 - q_1 \times m$. However,
according to the computations performed in steps 1 and 3,
$w = s^2 - q_1 \times m$. Therefore, we can conclude that $w = s^2 \bmod p$.


\HeadingLevelC{Analysis of Steps 5 - 8}

Similarly, steps $5 - 8$ in the algorithm check the correctness of $q_2$ and
use it to compute $w \times s \bmod m$. The key observation here is that
$q_2$ is the quotient of the integer division $(w \times s) / m$.

We can convince ourselves of the truth of this observation by using the fact
that $w = s^2 \bmod p$, which was proven above, by plugging in the definition
of the remainder in integer division, and by taking advantage of the
distributivity of integer multiplication with respect to addition.

\begin{align*}
\left\lfloor \frac{w \times s}{m} \right\rfloor
    & = \left\lfloor \frac{(s^2 \bmod m) \times s}{m} \right\rfloor \\
    & = \left\lfloor \frac{(s^2 - \lfloor \frac{s^2}{m} \rfloor \times m)
        \times s}{m} \right\rfloor \\
    & = \left\lfloor \frac{s^3 - \lfloor \frac{s^2}{m} \rfloor \times m
        \times s}{m} \right\rfloor \\
    & = \left\lfloor \frac{s^3 - q_1 \times m \times s}{m} \right\rfloor \\
    & = \left\lfloor \frac{s^3 - q_1 \times s \times m}{m} \right\rfloor \\
    & = q_2
\end{align*}

By the same argument used to analyze steps $1 - 4$, we use elementary division
properties to prove that $q_2$ is correct if and only if the equation below is
correct.

$$ 0 \le w \times s - q_2 \times m < m $$

The equation's first comparison, $0 \le w \times s - q_2 \times m$, is
equivalent to $q_2 \times m \le w \times s$, which corresponds to the check
performed by step 6. The second comparison, $w \times s - q_2 \times m < m$,
matches the condition verified by step 8. It follows that, if the algorithm
passes step 8, it must be the case that the value supplied for $q_2$ is
correct.

By plugging $w \times s$, $q_2$ and $m$ into the integer division remainder
definition, we obtain the identity
$w \times s \bmod m = w \times s - q_2 \times m$. Trivial substitution reveals
that the computations in steps 5 and 7 result in
$z = w \times s - q_2 \times m$, which allows us to conclude that
$z = w \times s \bmod m$.

In the analysis for steps $1 - 4$, we have proven that $w = s ^ 2 \bmod m$. By
substituting this into the above identity, we obtain the proof that the
algorithm's output is indeed the desired signed message.

\begin{align*}
z & = w \times s \bmod m \\
  & = (s^2 \bmod m) \times s \bmod m \\
  & = s^2 \times s \bmod m \\
  & = s^3 \bmod m
\end{align*}


\HeadingLevelC{Implementation Requirements}

The main advantage of the algorithm in Figure~\ref{fig:sgx_sig_verification} is
that it relies on the implementation of very few arithmetic operations on large
integers. The maximum integer size that needs to be handled is twice the size
of the modulus in the RSA key used to generate the signature.

Steps 1 and 5 use large integer multiplication. Steps 3 and 7 use integer
subtraction. Steps 2, 4, 6, and 8 use large integer comparison. The checks in
steps 2 and 6 guarantee that the results of the subtractions performed in steps
3 and 7 will be non-negative. It follows that the algorithm will never
encounter negative numbers.
