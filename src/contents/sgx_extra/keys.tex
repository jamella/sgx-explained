\subsection{Key Hierarchy and Derivation}

%% FUSE_KEY
%   US 8,972,746 B2 - Table 4-13, column 19

According to Intel's patents, the SGX implementation relies on a complex key
derivation process rooted on global secret keys in the CPU circuitry, and on
secrets embedded in the processor's eFUSEs. eFUSE information can be extracted
efficiently (Chipworks quoted us \$50-250k for extracting the entire eFUSE
contents from an Intel i5 processor), so some of the eFUSE secrets are
encrypted with a master key (referred to as a ``global wrapping logic key'' in
the patents).

The SGX patents describe two ``logic keys'' embedded in the CPU's circuitry,
which are the same for all CPUs in a stepping, making them essentially global
keys. The \textit{global wrapping logic key} (GWK) is a 128-bit AES key, and it is
used to encrypt a subset a 256-bit A.x value used to re-create the CPU's EPID
key, and a 128-bit \textit{pre-seed key 0}. The eFUSEs also contain a 128-bit
\textit{pre-seed key 1} and a 32-bit EPID group ID, which are stored in
cleartext.

\cite{shanbhogue2015gdxcsgx}, which describes mechanisms for proctecting SGX
keys in the presence of hardware debuggers, states that some hardware secrets
are stored in ``metal tie-ups and tie-downs'', while other secrets are stored
in e-fuses or in \textit{physically uncloneable function} (PUF) circuits.
\cite{brickell2014hardening, gotze2014provisioning} state that the GWK is
embedded into a chip using ``metal tie-ups and tie-downs'', and is shared by
all chips manufactured from the same mask set.

The SGX patents state that encrypting the eFUSE secrets by the logic key makes
them harder to extract via hardware monitoring tools, and protects them while
in transit to the CPU during the manufacturing process. This assumes that it is
very expensive to obtain the global key from a CPU, by virtue of the low
feature size. \cite{gotze2014provisioning} expresses concerns that the GWK can
be reverse-engineered.

\cite{gotze2014provisioning, gotze2014provisioning2} disclose that SGX
processors also employ a PUF, which generates a symmetric key that is used
during the provisioning process.  Specifically, at an early provisioning stage,
the PUF key is encrypted with the GWK and transmitted to the key generation
server. At a later stage, the key generation server encrypts the chip's fuse
key material with the PUF key, and transmits it to the chip. The PUF key
increases the cost of obtaining a chip's fuse key material, as an attacker must
compromise both provisioning stages in order to be able to decrypt the fuse key
material.

The SDM~\cite{intel2015sdm} mentions a 16-byte CPU security version number
(SVN), which contains the version numbers of various TCB components, and is a
source in the key derivation process. The patents further specify that the SVN
register is made up of (most likely 8-bit) sections that contain the SVNs of
each layer in the SGX initialization process, and that each initialization step
sets the corresponding section to its SVN, and then locks it for the duration
of the power-up cycle.

Intel's patents disclose that the key derivation process uses 128-bit AES in
ECB mode as a pseudo-random function (PRF). When an SVN is an input to a key
derivation process, a \textit{PRF loop} is used, where the PRF is applied to
a constant

\cite{anati2013sgx} confirms that the attestation uses Intel's
EPID~\cite{brickell2009epid} group signature scheme.

The Intel patents indicate that EREPORT's KeyID is initialized to a random
value on each processor power cycle, and is incremented after $2^{32}$ AES
operations that use the value. They also indicate that each EREPORT may
increment the KeyID by 1.
